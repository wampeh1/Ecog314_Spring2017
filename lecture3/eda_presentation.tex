\documentclass{beamer}
\usetheme{Antibes}
\usepackage{Sweave}
\begin{document}
%===============================================================================

\input{eda_presentation-concordance}
\title{Exploratory Data Analysis}

%---------------------------------------
\begin{frame}
\maketitle
\end{frame}


%---------------------------------------
\begin{frame}
\tableofcontents
\end{frame}


%---------------------------------------
\section{Introduction}

\begin{frame}
What is Exploratory Data Analysis?
\end{frame}


%---------------------------------------
\begin{frame}
\frametitle{What is EDA}
Exploratory Data Analysis is 

an approach to analyzing data sets to summarize their main characteristics, often with visual methods. (wikipedia \url{https://en.wikipedia.org/wiki/Exploratory_data_analysis})    

\end{frame}


%---------------------------------------
\begin{frame}
\frametitle{Goals of EDA}

The focus is on the data, not the model

\begin{itemize}
\item Learn about the data, underlying structure
\item Generate questions 
\item help decide what sort of model fits
\end{itemize}

\end{frame}



%---------------------------------------
\begin{frame}
\frametitle{What EDA is not}

Quantitative tests and models that impose assumptions (normality, etc.)
\begin{itemize}
\item Hypothesis testing, statistical inference
\item Model Specification (regressions, ANOVA)
\item Parameter estimation
\end{itemize}

\end{frame}

%---------------------------------------
\begin{frame}
\frametitle{What EDA is not}

Examples 
\begin{itemize}
\item 
\item 
\item 
\end{itemize}

\end{frame}


%---------------------------------------
\section{Techniques}
\begin{frame}

\begin{itemize}
\item 5-number summary
\item plots
\item 
\end{itemize}

\end{frame}


%===============================================================================
\end{document}



%---------------------------------------
\begin{frame}
\frametitle{}

\begin{itemize}
\item
\item
\item
\end{itemize}

\end{frame}
